This chapter shows some useful algorithms and implementations required to solve problems that require Dynamic Programming.

Some of the algorithms and implementations are as follows:

\lstinputlisting{../DP/dp1.cpp}

\lstinputlisting{../DP/dp2.cpp}

\lstinputlisting{../DP/dp3.cpp}

\lstinputlisting{../DP/dp4.cpp}

\lstinputlisting{../DP/dp5.cpp}

\lstinputlisting{../DP/dp6.cpp}

\lstinputlisting{../DP/dp7.cpp}

\lstinputlisting{../DP/dp8.cpp}

\section{Knapsack Problem}

The knapsack problem is a problem that consists of finding the maximum value of a set of items that can be placed in a knapsack of a given weight. The problem can be solved using Dynamic Programming.

The implementation can be done as follows:

\lstinputlisting{../DP/knapsack.cpp}

\lstinputlisting{../DP/knapsack2.cpp}

\section{Divide and Conquer}

The divide and conquer algorithm is a recursive algorithm that divides the problem into smaller subproblems and solves them recursively. The algorithm is as follows:

\begin{algorithm}
\caption{Divide and Conquer}
\label{alg:divideandconquer}
\begin{algorithmic}[1]
\Procedure{DivideAndConquer}{$A$}
\If{$A$ has only one element}
\State return $A$
\EndIf
\State $B \gets DivideAndConquer($A[0..n/2]$)$
\State $C \gets DivideAndConquer($A[n/2+1..n]$)$
\State return $Merge(B, C)$
\EndProcedure
\end{algorithmic}
\end{algorithm}

The implementation can be done as follows:

\lstinputlisting{../DP/DivideAndConquer.cpp}

\lstinputlisting{../DP/DivideAndConquer2.cpp}

\section{Digit DP}

Digit DP is a technique that can be used to solve problems that require Dynamic Programming. The technique consists of solving the problem by using Dynamic Programming and the digits of the number.

The implementation can be done as follows:

\lstinputlisting{../DP/digitdp.cpp}

\section{Alien's Trick}

The Alien's Trick is a technique that can be used to solve problems that require Dynamic Programming. The technique consists of solving the problem by using Dynamic Programming and the digits of the number.

The implementation can be done as follows:

\lstinputlisting{../DP/alienstrick.cpp}

\section{Bitwise Digit DP}

Bitwise Digit DP is a technique that can be used to solve problems that require Dynamic Programming. The technique consists of solving the problem by using Dynamic Programming and the digits of the number.

The implementation can be done as follows:

\lstinputlisting{../DP/bitwisedigitdp.cpp}

\section{Broken Profile}

The Broken Profile is a technique that can be used to solve problems that require Dynamic Programming. The technique consists of solving the problem by using Dynamic Programming and the digits of the number.

The implementation can be done as follows:

\lstinputlisting{../DP/BrokenProfile.cpp}

\section{CHT}

The Convex Hull Trick is a technique that can be used to solve problems that require Dynamic Programming. The technique consists of solving the problem by using Dynamic Programming and the digits of the number.

The implementation can be done as follows:

\lstinputlisting{../DP/CHT.cpp}

A dynamical implementation for the CHT can be done as follows:

\lstinputlisting{../DP/CHT2.cpp}


\section{Exchange Arguments}

The Exchange Arguments is a technique that can be used to solve problems that require Dynamic Programming. The technique consists of solving the problem by using Dynamic Programming and the digits of the number.

The implementation can be done as follows:

\lstinputlisting{../DP/ExchangeArguments.cpp}


\section{Expected Value}

The Expected Value is a technique that can be used to solve problems that require Dynamic Programming. The technique consists of solving the problem by using Dynamic Programming and the digits of the number.

The implementation can be done as follows:

\lstinputlisting{../DP/ExpectedValue.cpp}

\section{Largest Sum Contiguous Subarray}

The Largest Sum Contiguous Subarray is a technique that can be used to solve problems that require Dynamic Programming. The technique consists of solving the problem by using Dynamic Programming and the digits of the number.

The implementation can be done as follows:

\lstinputlisting{../DP/LargestSumContiguousSubarray.cpp}

