This chapter is about some useful mathematical tools needed in order to solve problems.

\section{GCD and LCM}

In order to find the greatest common divisor (GCD) of two numbers, the Euclidean algorithm can be used. The implementation is as follows:

\lstinputlisting{Mathematics/Euclid.cpp}

Another (and faster) way to find the GCD is by using the following code:

\lstinputlisting{Mathematics/FastGCD.cpp}

The way Halim suggests to find the GCD and the LCM is given by the following code:

\lstinputlisting{Mathematics/HalimGCD.cpp}

\section{Prime Numbers}

The fastest way to check the primality of a number is by using Erathostenes' sieve. The typical implementation is as follows:

\lstinputlisting{Mathematics/Erathostenes.cpp}

Nevertheless, the following implementation is faster, since  the statement if \begin{verbatim}
if (i % prime[j] == 0) break;
\end{verbatim}
 terminates the loop when p divides i. The inner loop is executed only once for each composite. Hence, the code performs in O(n) complexity, resulting in the 'linear' sieve:

\lstinputlisting{Mathematics/LinearSieve.cpp}


%\VerbatimInput{Graphs/BFS.cpp}
