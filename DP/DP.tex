This chapter shows some useful algorithms and implementations required to solve problems that require Dynamic Programming.

Some of the algorithms and implementations are as follows:

\lstinputlisting{../DP/dp1.cpp}

\lstinputlisting{../DP/dp2.cpp}

\lstinputlisting{../DP/dp3.cpp}

\lstinputlisting{../DP/dp4.cpp}

\lstinputlisting{../DP/dp5.cpp}

\lstinputlisting{../DP/dp6.cpp}

\lstinputlisting{../DP/dp7.cpp}

\lstinputlisting{../DP/dp8.cpp}

\section{Knapsack Problem}

The knapsack problem is a problem that consists of finding the maximum value of a set of items that can be placed in a knapsack of a given weight. The problem can be solved using Dynamic Programming.

The implementation can be done as follows:

\lstinputlisting{../DP/knapsack.cpp}

\section{Divide and Conquer}

The divide and conquer algorithm is a recursive algorithm that divides the problem into smaller subproblems and solves them recursively. The algorithm is as follows:

\begin{algorithm}
\caption{Divide and Conquer}
\label{alg:divideandconquer}
\begin{algorithmic}[1]
\Procedure{DivideAndConquer}{$A$}
\If{$A$ has only one element}
\State return $A$
\EndIf
\State $B \gets DivideAndConquer($A[0..n/2]$)$
\State $C \gets DivideAndConquer($A[n/2+1..n]$)$
\State return $Merge(B, C)$
\EndProcedure
\end{algorithmic}
\end{algorithm}

The implementation can be done as follows:

\lstinputlisting{../DP/DivideAndConquer.cpp}
